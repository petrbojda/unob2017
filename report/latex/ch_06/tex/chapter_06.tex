% ********************************** CHAPTER 5 source ***********************************************
\section{Implementace - popis vytvořených programů}

Implementace algoritmů do SDR byly vytvořeny v jazyce Python, variantě 2.7. s využitím nástrojů balíku GNU Radio verze 3.7.10.1, dále knihoven NumPy 1.11.1 a SciPy 0.18.1. Simulace byly napsány, vyzkoušeny a provozovány na počítači s operačním systémem Linux Ubuntu 16.04. 

Dále pro přímou kompilaci algoritmů napsaných v jazyce C++ a využívajících API funkcí dodaných s ovladačem UHD výrobce rádia Ettus E310 bylo použito kompilátoru GNU C++ verze 4.9.2. Verze ovladače UHD 003.009.002.

Vytvořené příklady implementací jsou umístěné v podadresářích uvnitř samostatného adresáře \texttt{srcpy\\hwimpl}, resp. \texttt{srcc}. Členění do podadresářů je podle logické a funkční příslušnosti. 

Funkce ze souborů ve vnořených adresářích lze využít až poté, co jsou do skriptu hlavní úrovně importovány\footnote{Toto je pochopitelně dokumentovanou vlastností jazyka Python. Zde je tento fakt zmíněn pouze jako připomínka.}. 

\subsection{Implementace vysílače pomocí GNURadio v jazyce Python}

Funkce zde popsané jsou umístěny v podadresáři \texttt{siggens}. Patří k základním prostředkům modulací -- generují signál požadovaného tvaru a parametrů a to jak v podobě základní (jediný, neopakující se impulz), tak i ve variantě opakujících se impulzů obsahujících kód. 



\subsection{Implementace vysílače pomocí API funkcí UHD v jazyce C++}

Programy, které byly vytvořeny proto, aby demonstrovaly možnosti použití API funkcí dodaných výrobcem rádia spolu s ovladači UHD jsou uloženy v adresáři \texttt{srcc}.

Jsou napsány jako zdrojové soubory jazyka c++ v souborech s koncovkou \texttt{ .cpp}. Definice použitých tříd jsou potom přidány v podobě hlavičkových souborů s koncovkou \texttt{ .hpp}.

Pro to, aby bylo možno programy spustit na cílové platformě, je potřeba je tam i zkompilovat. Spolu s API funkcemi výrobce dodává i vlastní sadu příkladů použití a dále vzorový příklad kompilace. Ten byl po nezbytných úpravách uložen v adresáři \texttt{src\\UHD\_compile}. Postup je popsán v přiloženém souboru \texttt{README}.


